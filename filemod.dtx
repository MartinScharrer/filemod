% \iffalse meta-comment
%
% Copyright (C) 2011 by Martin Scharrer <martin@scharrer-online.de>
% ---------------------------------------------------------------------------
% This work may be distributed and/or modified under the
% conditions of the LaTeX Project Public License, either version 1.3
% of this license or (at your option) any later version.
% The latest version of this license is in
%   http://www.latex-project.org/lppl.txt
% and version 1.3 or later is part of all distributions of LaTeX
% version 2005/12/01 or later.
%
% This work has the LPPL maintenance status `maintained'.
%
% The Current Maintainer of this work is Martin Scharrer.
%
% This work consists of the files filemod.dtx and filemod.ins
% and the derived filebase filemod.sty.
%
% \fi
%
% \iffalse
%<*driver>
\ProvidesFile{filemod.dtx}
%</driver>
%<package>\NeedsTeXFormat{LaTeX2e}[1999/12/01]
%<package>\ProvidesPackage{filemod}
%<*package>
    [2011/03/09 v1.0 Get and compare file modification times]
%</package>
%
%<*driver>
\documentclass{ydoc}
\usepackage{filemod}[2011/03/09]
\MakeShortMacroArgs\`\relax
\EnableCrossrefs
\CodelineIndex
\RecordChanges
\begin{document}
  \DocInput{filemod.dtx}
  \PrintChanges
  \PrintIndex
\end{document}
%</driver>
% \fi
%
% \CheckSum{555}
%
% \CharacterTable
%  {Upper-case    \A\B\C\D\E\F\G\H\I\J\K\L\M\N\O\P\Q\R\S\T\U\V\W\X\Y\Z
%   Lower-case    \a\b\c\d\e\f\g\h\i\j\k\l\m\n\o\p\q\r\s\t\u\v\w\x\y\z
%   Digits        \0\1\2\3\4\5\6\7\8\9
%   Exclamation   \!     Double quote  \"     Hash (number) \#
%   Dollar        \$     Percent       \%     Ampersand     \&
%   Acute accent  \'     Left paren    \(     Right paren   \)
%   Asterisk      \*     Plus          \+     Comma         \,
%   Minus         \-     Point         \.     Solidus       \/
%   Colon         \:     Semicolon     \;     Less than     \<
%   Equals        \=     Greater than  \>     Question mark \?
%   Commercial at \@     Left bracket  \[     Backslash     \\
%   Right bracket \]     Circumflex    \^     Underscore    \_
%   Grave accent  \`     Left brace    \{     Vertical bar  \|
%   Right brace   \}     Tilde         \~}
%
%
% \changes{v1.0}{2011/03/09}{Converted to DTX file}
%
% \DoNotIndex{\newcommand,\newenvironment}
%
% \providecommand*{\url}{\texttt}
% \GetFileInfo{filemod.dtx}
% \title{The \textsf{filemod} package}
% \author{Martin Scharrer \\ \url{martin@scharrer-online.de}}
% \date{\fileversion~from \filedate}
%
% \maketitle
%
% \section{Introduction}
%
%
% \section{Usage}
%
% \DescribeMacro{\filemodparse}{<macro>}{<filename>}
% Parses the file modification datetime of the given file and passes the result to the given macro.
% The macro will receive seven arguments:
% \begin{quote}
% `<macro>{<YYYY>}{<MM>}{<DD>}{<HH>}{<mm>}{<SS>}{<TZ>}`
% \end{quote}
% i.e.\ year, month, day, hour, minutes, seconds and the timezone as signed offset
% or `Z` (catcode 12).
%
%
% \DescribeMacro{\filemodnotexists}{<macro>}
% This macro will be called by `|\filemodparse` with the original given macro when the given file does not exists. By default
% it contains all zeros except `Z` (catcode 12) as timezone:
%
% \begin{quote}
% `!\#1!{0000}{00}{00}{00}{00}{00}{Z}`
% \end{quote}
% The user can redefine this macro to a different content, e.g.\ to a different fall-back value or to display a warning.
% Note if this macro contains non-expandable code the macros which uses it aren't expandable anymore.
%
%
%
%
% \subsection{Print File Modification Date and Time}
%
% \DescribeMacro{\filemodprint}{<filename>}
% Prints the file modifications date and time by calling `|\filemodparse|\thefilemod{<filename>}`.
%
%
% \DescribeMacro{\filemodprintdate}{<filename>}
% Prints the file modifications date by calling `|\filemodparse|\thefilemoddate{<filename>}`.
%
%
% \DescribeMacro{\filemodprinttime}{<filename>}
% Prints the file modifications time by calling `|\filemodparse|\thefilemodtime{<filename>}`.
%
%
% \DescribeMacro{\thefilemod}
% Reads the date and time as seven arguments and typesets it. This macro can be redefined to a custom format.\\
% Default format: '\filemodprint{\jobname.dtx}'
%
%
% \DescribeMacro{\thefilemoddate}
% Reads the date and time as seven arguments and typesets only the date. This macro can be redefined to a custom format.\\
% Default format: '\filemodprintdate{\jobname.dtx}'
%
%
% \DescribeMacro{\thefilemodtime}
% Reads the date and time as seven arguments and typesets only the time. This macro can be redefined to a custom format.\\
% Default format: '\filemodprinttime{\jobname.dtx}'
%
%
% \subsection{Get File Modification Date and Time as Number}
%
% \DescribeMacro{\filemodnumdate}{<filename>}
% Expands to an integer of the form |YYYYMMDD| which can be used for numeric comparisons like `|\ifnum`.
% This macros uses `|\filemodparse` and `|\filemodnotexists` will be used if the file does not exist.
%
% \DescribeMacro{\filemodnumtime}{<filename>}
% Expands to an integer of the form |HHmmSS| which can be used for numeric comparisons like `|\ifnum`.
% This macros uses `|\filemodparse` and `|\filemodnotexists` will be used if the file does not exist.
%
% \DescribeMacro{\filemodNumdate}{<filename>}
% Expands to an integer of the form |YYYYMMDD| which can be used for numeric comparisons like `|\ifnum`.
% Parses the file modification date by itself and will return |00000000| if the file does not exist.
%
% \DescribeMacro{\filemodNumtime}{<filename>}
% Expands to an integer of the form |HHmmSS| which can be used for numeric comparisons like `|\ifnum`.
% Parses the file modification date by itself and will return |000000| if the file does not exist.
%
%
% \DescribeMacro{\filemodgetnum}{<filename>}
% Stores the file modification date and time as numbers (|YYYYMMDD| and |HHmmSS|) as well the timezone string
% into the macros `|\filemoddate`, `|\filemodtime` and `|\filemodtz`.
%
%
% \subsection{Compare File Modification Date/Time}
%
% \DescribeMacro{\filemodnewest}[<num>]{<filename 1>}{<filename 2>}
% Expands the file name of the newest given file or file `<num>` if both file modification dates are identical.
%
% \DescribeMacro{\filemodoldest}[<num>]{<filename 1>}{<filename 2>}
% Expands the file name of the oldest given file or file `<num>` if both file modification dates are identical.
%
%
% \DescribeMacro{\filemodNewest}[<num>]{{<filename 1>}{<filename 2>}'...'{<filename n>}}
% Expands the file name of the newest given file.
% The files are compared in pairs of two in the given order (i.e.\ first 1 and 2 and the result with 3 etc.)
% The optional argument `<num>` can be used to indicate which file name should be used if both file modification dates are identical.
%
%
% \DescribeMacro{\filemodOldest}[<num>]{{<filename 1>}{<filename 2>}'...'{<filename n>}}
% Expands the file name of the oldest given file.
% The files are compared in pairs of two in the given order (i.e.\ first 1 and 2 and the result with 3 etc.)
% The optional argument `<num>` can be used to indicate which file name should be used if both file modification dates are identical.
%
%
% \DescribeMacro{\Filemodnewest}[<num>]{<filename 1>}{<filename 2>}
% Same as `|\filemodnewest` just not expandable but faster.
%
%
% \DescribeMacro{\Filemodoldest}[<num>]{<filename 1>}{<filename 2>}
% Same as `|\filemodoldest` just not expandable but faster.
%
%
% \DescribeMacro{\FilemodNewest}[<num>]{{<filename 1>}{<filename 2>}'...'{<filename n>}}
% Same as `|\filemodNewest` just not expandable but faster.
%
%
% \DescribeMacro{\FilemodOldest}[<num>]{{<filename 1>}{<filename 2>}'...'{<filename n>}}
% Same as `|\filemodOldest` just not expandable but faster.
%
%
% \def\optional{\color{black!25}\colorlet{meta}{meta!50!black!25}}
% \DescribeMacro{\filemodcmp}[<num>]{<filename 1>}{<filename 2>}{<clause 1>}{<clause 2>}!\optional!{<clause 3>}
% This macro compares the file modification date and time of the two given files and expands to the clause of the
% newest file. An numerical optional argument can be given to determine the outcome if both files have the exact same 
% modification date/time (or both do not exists). If `<num>` is 0, no clause will be expanded, i.e.\ the macro expands
% to an empty text. If `<num>` is 1 (default) or 2 the macro expands to the corresponding clause.
% However if `<num>` is 3, the macro will await a third clause and expands to it if both files modification dates are equal.
%
% This macro is fully expandable even when the optional argument is used. However, `<filename 1>` must not be equal to "[".
%
%
% \DescribeMacro{\filemodCmp}{<filename 1>}{<filename 2>}{<clause 1>}{<clause 2>}
% This is a simpler and therefore faster version of `|\filemodcmp`. It is fully expandable, does not take any optional
% arguments and will always expand to the first clause if both file modification dates are equal (or both files do not exist).
% The `|\filemodNumdate` and `|\filemodNumtime` macros are used in the comparison.
%
%
% \DescribeMacro{\Filemodcmp}[<num>]{<filename 1>}{<filename 2>}{<clause 1>}{<clause 2>}!\optional!{<clause 3>}
% This macro provides the same functionality as `|\filemodcmp`.
% It is not expandable but will be processed faster. The optional
% argument is processed like normally.
%
% \DescribeMacro{\FilemodCmp}[<num>]{<filename 1>}{<filename 2>}
% This macro will compare the two file modification dates like `|\Filemodcmp` and `|\filemodcmp` but does not take
% the possible clauses as arguments, instead it stores the result into the expandable macro `|\filemodcmpresult`
% which then takes `{<clause 1>}{<clause 2>}` (and also `{<clause 3>}` if `<num>` was 3) as arguments and expand to the one
% corresponding to the newest file.
% This set of macros gives the user the speed benefit of `|\Filemodcmp` while still be able to use the result in an expandable context.
%
%
% \DescribeMacro{\filemodcmpdefault}
% Holds the default number (i.e.\ 1) for the optional arguments of the previous mentioned macros.
% This macro can be redefined with a number or a numeric expression valid for |\ifcase|. It should not contain any
% trailing spaces.
%
%
% \DescribeMacro{\filemodZ}
% Defined to "Z" with catcode 12 as it is returned as timezone.
% This might be useful for comparisons or custom definitions.
%
% \DescribeMacro{\filemodz}
% Let (`|\let`) to "Z" with catcode 12 as it is returned as timezone.
% This might be useful for comparisons or custom definitions.
%
%
% \StopEventually{}
%
% \section{Implementation}
%
% \iffalse
%<*package>
% \fi
%
%
% \begin{macro}{\filemodparse}
%%%% BASIC PARSER
%    \begin{macrocode}
\newcommand*\filemodparse[2]{%
    \expandafter\filemod@parse\pdffilemoddate{#2}\relax{#1}%
}
%    \end{macrocode}
% \end{macro}
%
%
% \begin{macro}{\filemod@parse}
%    \begin{macrocode}
\def\filemod@parse#1\relax#2{%
    \ifx\relax#1\relax
        \expandafter\@firstoftwo
    \else
        \expandafter\@secondoftwo
    \fi
    {\filemodnotexists{#2}}%
    {\filemod@parse@#1\empty{#2}\relax}%
}
%    \end{macrocode}
% \end{macro}
%
%    \begin{macrocode}
\begingroup
\@makeother\D
\@makeother\Z
\@makeother:
%    \end{macrocode}
%
% \begin{macro}{\filemod@parse@D:}
%    \begin{macrocode}
\gdef\filemod@parse@D:#1#2#3#4#5#6#7#8#9\relax{%
    \filemod@parse@@{{#1#2#3#4}{#5#6}{#7#8}}#9\relax
}
%    \end{macrocode}
% \end{macro}
%
%
% \begin{macro}{\filemodnotexists}
%    \begin{macrocode}
\gdef\filemodnotexists#1{%
    #1{0000}{00}{00}{00}{00}{00}{Z}%
}
%    \end{macrocode}
% \end{macro}
%
%
% \begin{macro}{\filemodNumdate}
%    \begin{macrocode}
\gdef\filemodNumdate#1{%
    \expandafter\filemod@Numdate\pdffilemoddate{#1}D:00000000000000Z\relax
}
%    \end{macrocode}
% \end{macro}
%
%
% \begin{macro}{\filemod@Numdate}
%    \begin{macrocode}
\gdef\filemod@NumdateD:#1#2#3#4#5#6#7#8#9\relax{%
    #1#2#3#4#5#6#7#8%
}
%    \end{macrocode}
% \end{macro}
%
%
% \begin{macro}{\filemodNumtime}
%    \begin{macrocode}
\gdef\filemodNumtime#1{%
    \expandafter\filemod@Numtime\pdffilemoddate{#1}D:00000000000000Z\relax
}
%    \end{macrocode}
% \end{macro}
%
%
% \begin{macro}{\filemod@Numtime}
%    \begin{macrocode}
\gdef\filemod@NumtimeD:#1#2#3#4#5#6#7#8#9\relax{%
    \filemod@@Numtime#9\relax
}
%    \end{macrocode}
% \end{macro}
%
%
% \begin{macro}{\filemod@@Numtime}
%    \begin{macrocode}
\gdef\filemod@@Numtime#1#2#3#4#5#6#7\relax{%
    #1#2#3#4#5#6%
}
%    \end{macrocode}
% \end{macro}
%
% \begin{macro}{\filemodZ}
%    \begin{macrocode}
\expandafter\gdef\csname filemodZ\endcsname{Z}%
\let\filemodz=Z\relax
%    \end{macrocode}
% \end{macro}
%
%    \begin{macrocode}
\endgroup
%    \end{macrocode}
%
% \begin{macro}{\filemod@parse@@}
%    \begin{macrocode}
\def\filemod@parse@@#1#2#3#4#5#6#7#8\empty#9\relax{%
    #9#1{#2#3}{#4#5}{#6#7}{#8}%
}
%    \end{macrocode}
% \end{macro}
%
%%%%%
%
%
% \begin{macro}{\filemodprint}
%    \begin{macrocode}
\newcommand*\filemodprint{\filemodparse\thefilemod}
%    \end{macrocode}
% \end{macro}
%
%
% \begin{macro}{\filemodprintdate}
%    \begin{macrocode}
\newcommand*\filemodprintdate{\filemodparse\the@filemoddate}
%    \end{macrocode}
% \end{macro}
%
%
% \begin{macro}{\filemodprinttime}
%    \begin{macrocode}
\newcommand*\filemodprinttime{\filemodparse\the@filemodtime}
%    \end{macrocode}
% \end{macro}
%
%
% \begin{macro}{\thefilemod}
%    \begin{macrocode}
\newcommand*\thefilemod[7]{%
    \thefilemoddate{#1}{#2}{#3}%
    \filemodsep
    \thefilemodtime{#4}{#5}{#6}{#7}%
}
%    \end{macrocode}
% \end{macro}
%
%    \begin{macrocode}
\let\filemodsep\space
%    \end{macrocode}
%
% \begin{macro}{\thefilemoddate}
%    \begin{macrocode}
\newcommand*\thefilemoddate[3]{%
    #1/#2/#3%
}
%    \end{macrocode}
% \end{macro}
%
%
% \begin{macro}{\thefilemodtime}
%    \begin{macrocode}
\newcommand*\thefilemodtime[4]{%
    #1:#2:#3~#4%
}
%    \end{macrocode}
% \end{macro}
%
%
% \begin{macro}{\the@filemoddate}
%    \begin{macrocode}
\def\the@filemoddate#1#2#3#4#5#6#7{%
    \thefilemoddate{#1}{#2}{#3}%
}
%    \end{macrocode}
% \end{macro}
%
%
% \begin{macro}{\the@filemodtime}
%    \begin{macrocode}
\def\the@filemodtime#1#2#3{%
    \thefilemodtime
}
%    \end{macrocode}
% \end{macro}
%
%%%%%%%
%
%
% \begin{macro}{\filemodnumdate}
%%%% NUMERIC RESULTS
%    \begin{macrocode}
\newcommand*\filemodnumdate{\filemodparse\filemod@numdate}
%    \end{macrocode}
% \end{macro}
%
%
% \begin{macro}{\filemod@numdate}
%    \begin{macrocode}
\def\filemod@numdate#1#2#3#4#5#6#7{#1#2#3}
%    \end{macrocode}
% \end{macro}
%
%
% \begin{macro}{\filemodnumtime}
%    \begin{macrocode}
\newcommand*\filemodnumtime{\filemodparse\filemod@numtime}
%    \end{macrocode}
% \end{macro}
%
%
% \begin{macro}{\filemod@numtime}
%    \begin{macrocode}
\def\filemod@numtime#1#2#3#4#5#6#7{#4#5#6}
%    \end{macrocode}
% \end{macro}
%
%
% \begin{macro}{\filemodgetnum}
%    \begin{macrocode}
\newcommand*\filemodgetnum{\filemodparse\filemod@getnum}
%    \end{macrocode}
% \end{macro}
%
%
% \begin{macro}{\filemod@getnum}
%    \begin{macrocode}
\def\filemod@getnum#1#2#3#4#5#6#7{%
    \def\filemoddate{#1#2#3}%
    \def\filemodtime{#4#5#6}%
    \def\filemodtz{#7}%
}
%    \end{macrocode}
% \end{macro}
%
%
%
% \subsection{ADVANCED FILE DATA COMPARISON}
%
%
% \begin{macro}{\filemod@newest}
%    \begin{macrocode}
\def\filemod@newest#1#2#3{%
    \filemod@@cmp>{#1}{#2}{#3}{#2}{#3}%
}
%    \end{macrocode}
% \end{macro}
%
%
% \begin{macro}{\filemod@oldest}
%    \begin{macrocode}
\def\filemod@oldest#1#2#3{%
    \filemod@@cmp<{#1}{#2}{#3}{#2}{#3}%
}
%    \end{macrocode}
% \end{macro}
%
%
% \begin{macro}{\filemodnewest}
% Expandable form
%    \begin{macrocode}
\newcommand*\filemodnewest{%
    \filemod@opt\filemod@newest@opt\filemod@newest
}
%    \end{macrocode}
% \end{macro}
%
%
% \begin{macro}{\filemod@opt}
%    \begin{macrocode}
\def\filemod@opt#1#2#3{%
    \expandafter
    \remove@to@nnil@exec
    \ifx[#3\@nnil\remove@to@nnil
      \expandafter#1%
    \else\@nnil\empty
      \expandafter#2%
      \expandafter\filemodcmpdefault
    \fi
    {#3}%
}
%    \end{macrocode}
% \end{macro}
%
%
% \begin{macro}{\remove@to@nnil@exec}
%    \begin{macrocode}
\def\remove@to@nnil@exec#1\@nnil#2{%
    \ifx\@nnil#1\@nnil\else
      \expandafter#2
    \fi
}
%    \end{macrocode}
% \end{macro}
%
%
% \begin{macro}{\filemod@newest@opt}
%    \begin{macrocode}
\def\filemod@newest@opt#1#2]{%
    \filemod@newest{#2}%
}
%    \end{macrocode}
% \end{macro}
%
%
% \begin{macro}{\filemodoldest}
%    \begin{macrocode}
\newcommand*\filemodoldest{%
    \filemod@opt\filemod@oldest@opt\filemod@oldest
}
%    \end{macrocode}
% \end{macro}
%
%
% \begin{macro}{\filemod@oldest@opt}
%    \begin{macrocode}
\def\filemod@oldest@opt#1#2]{%
    \filemod@oldest{#2}%
}
%    \end{macrocode}
% \end{macro}
%
%%%%%%%  Expandable, groups
%
%
% \begin{macro}{\filemodNewest}
%    \begin{macrocode}
\newcommand*\filemodNewest{}
\def\filemodNewest#1#{%
  \expandafter\expandafter
  \expandafter\@filemodNewest
  \csname
    @%
  \ifx\@nnil#1\@nnil
    first%
  \else
    second%
  \fi
    oftwo%
  \endcsname
    {[\filemodcmpdefault]}%
    {#1}%
}
%    \end{macrocode}
% \end{macro}
%
%
% \begin{macro}{\@filemodNewest}
%    \begin{macrocode}
\def\@filemodNewest[#1]#2{%
    \@@filemodNewest{#1}#2\filemod@end
}
%    \end{macrocode}
% \end{macro}
%
%
% \begin{macro}{\@@filemodNewest}
%    \begin{macrocode}
\def\@@filemodNewest#1#2{%
    \filemod@Newest{#2}{#1}%
}
%    \end{macrocode}
% \end{macro}
%
%
% \begin{macro}{\filemodOldest}
%    \begin{macrocode}
\newcommand*\filemodOldest{}
\def\filemodOldest#1#{%
  \expandafter\expandafter
  \expandafter\@filemodOldest
  \csname
    @%
  \ifx\@nnil#1\@nnil
    first%
  \else
    second%
  \fi
    oftwo%
  \endcsname
    {[\filemodcmpdefault]}%
    {#1}%
}
%    \end{macrocode}
% \end{macro}
%
%
% \begin{macro}{\@filemodOldest}
%    \begin{macrocode}
\def\@filemodOldest[#1]#2{%
    \@@filemodOldest{#1}#2\filemod@end
}
%    \end{macrocode}
% \end{macro}
%
%
% \begin{macro}{\@@filemodOldest}
%    \begin{macrocode}
\def\@@filemodOldest#1#2{%
    \filemod@Oldest{#2}{#1}%
}
%    \end{macrocode}
% \end{macro}
%
%
% \begin{macro}{\filemod@Newest}
%    \begin{macrocode}
\def\filemod@Newest#1#2#3{%
  \iffilemod@end{#3}%
    {#1}%
    {%
    \expandafter\expandafter
    \expandafter\expandafter
    \expandafter\expandafter
    \expandafter\filemod@Newest
    \expandafter\expandafter
    \expandafter\expandafter
    \expandafter\expandafter
    \expandafter{%
    \expandafter\expandafter
    \expandafter\@gobble
    \expandafter\string\csname\filemod@@cmp>{#2}{#1}{#3}{#1}{#3}\endcsname}{#2}}%
}
%    \end{macrocode}
% \end{macro}
%
%
% \begin{macro}{\filemod@Oldest}
%    \begin{macrocode}
\def\filemod@Oldest#1#2#3{%
  \iffilemod@end{#3}%
    {#1}%
    {%
    \expandafter\expandafter
    \expandafter\expandafter
    \expandafter\expandafter
    \expandafter\filemod@Oldest
    \expandafter\expandafter
    \expandafter\expandafter
    \expandafter\expandafter
    \expandafter{%
    \expandafter\expandafter
    \expandafter\@gobble
    \expandafter\string\csname\filemod@@cmp<{#2}{#1}{#3}{#1}{#3}\endcsname}{#2}}%
}
%    \end{macrocode}
% \end{macro}
%
%
% \begin{macro}{\iffilemod@end}
%    \begin{macrocode}
\def\iffilemod@end#1{%
  \ifx\filemod@end#1%
    \expandafter\@firstoftwo
  \else
    \expandafter\@secondoftwo
  \fi
}
%    \end{macrocode}
% \end{macro}
%
%
% \begin{macro}{\filemod@end}
%    \begin{macrocode}
\def\filemod@end{\@gobble{filemod@end}}
%    \end{macrocode}
% \end{macro}
%
%%%%% BASIC FILE DATA COMPARISON
%
%
% \begin{macro}{\filemodCmp}
% Expandable form
%    \begin{macrocode}
\newcommand*\filemodCmp[2]{%
    \ifcase0%
        \ifnum\filemodNumdate{#2}>\filemodNumdate{#1}\space1\else
            \ifnum\filemodNumdate{#2}=\filemodNumdate{#1}\space
                \ifnum\filemodNumtime{#2}>\filemodNumtime{#1}\space1\fi
            \fi
        \fi
    \space
       \expandafter\@firstoftwo
    \or
       \expandafter\@secondoftwo
    \fi
}
%    \end{macrocode}
% \end{macro}
%
%
% \begin{macro}{\filemodcmp}
% Expandable form
%    \begin{macrocode}
\newcommand*\filemodcmp{%
    \filemod@opt\filemod@cmp@opt\filemod@cmp
}
%    \end{macrocode}
% \end{macro}
%
%
% \begin{macro}{\filemodcmpdefault}
%    \begin{macrocode}
\newcommand*\filemodcmpdefault{1}
%    \end{macrocode}
% \end{macro}
%
%
% \begin{macro}{\filemod@cmp@opt}
%    \begin{macrocode}
\def\filemod@cmp@opt#1#2]{%
    \filemod@cmp{#2}%
}
%    \end{macrocode}
% \end{macro}
%
%
% \begin{macro}{\filemod@cmp}
%    \begin{macrocode}
\def\filemod@cmp{\filemod@@cmp>}
%    \end{macrocode}
% \end{macro}
%
%
% \begin{macro}{\filemod@@cmp}
%    \begin{macrocode}
\def\filemod@@cmp#1#2#3#4{%
    \ifcase0%
        \ifnum\filemodnumdate{#4}#1\filemodnumdate{#3}\space1\else
            \ifnum\filemodnumdate{#4}=\filemodnumdate{#3}\space
                \ifnum\filemodnumtime{#4}#1\filemodnumtime{#3}\space1\else
                    \ifnum\filemodnumdate{#4}=\filemodnumdate{#3}\space2\fi
                \fi
            \fi
        \fi
    \space
       \csname\ifnum#1>2 @firstofthree\else @firstoftwo\fi\expandafter\endcsname
    \or
       \csname\ifnum#2>2 @secondofthree\else @secondoftwo\fi\expandafter\endcsname
    \else
       \csname
       \ifcase#2
         @gobbletwo%
       \or
         @firstoftwo%
       \or
         @secondoftwo%
       \else
         @thirdofthree%
       \fi
       \expandafter
       \endcsname
    \fi
}
%    \end{macrocode}
% \end{macro}
%
%
% \begin{macro}{\@firstofthree}
%    \begin{macrocode}
\long\def\@firstofthree#1#2#3{#1}
%    \end{macrocode}
% \end{macro}
%
%
% \begin{macro}{\@secondofthree}
%    \begin{macrocode}
\long\def\@secondofthree#1#2#3{#2}
%    \end{macrocode}
% \end{macro}
%
%
% \begin{macro}{\Filemodnewest}
%    \begin{macrocode}
\newcommand*\Filemodnewest[2]{\FilemodNewest{{#1}{#2}}}
%    \end{macrocode}
% \end{macro}
%
%
% \begin{macro}{\Filemodoldest}
%    \begin{macrocode}
\newcommand*\Filemodoldest[2]{\FilemodOldest{{#1}{#2}}}
%    \end{macrocode}
% \end{macro}
%
%
% \begin{macro}{\FilemodOldest}
%    \begin{macrocode}
\newcommand*\FilemodOldest[2][1]{%
    \def\filemode@tie{#1}%
    \def\filemod@gl{<}%
    \Filemod@Newest#2\filemod@end
}
%    \end{macrocode}
% \end{macro}
%
%
% \begin{macro}{\FilemodNewest}
%    \begin{macrocode}
\newcommand*\FilemodNewest[2][1]{%
    \def\filemode@tie{#1}%
    \def\filemod@gl{>}%
    \Filemod@Newest#2\filemod@end
}
%    \end{macrocode}
% \end{macro}
%
%
% \begin{macro}{\Filemod@Newest}
%    \begin{macrocode}
\def\Filemod@Newest#1{%
    \def\filemodcmpfile{#1}%
    \filemodgetnum{#1}%
    \let\filemodcmpdate\filemoddate
    \let\filemodcmptime\filemodtime
    \Filemod@@Newest
}
%    \end{macrocode}
% \end{macro}
%
%
% \begin{macro}{\Filemod@@Newest}
%    \begin{macrocode}
\def\Filemod@@Newest#1{%
  \iffilemod@end{#1}{}{%
    \filemodgetnum{#1}%
    \ifcase0%
        \ifnum\filemoddate\filemod@gl\filemodcmpdate\space1\else
            \ifnum\filemoddate=\filemodcmpdate\space
                \ifnum\filemodtime\filemod@gl\filemodcmptime\space1\else
                    \ifnum\filemoddate=\filemodcmpdate\space
                        \ifnum\filemode@tie=1\else 1\fi
                    \fi
                \fi
            \fi
        \fi
    \else
        \def\filemodcmpfile{#1}%
        \let\filemodcmpdate\filemoddate
        \let\filemodcmptime\filemodtime
    \fi
    \Filemod@@Newest
  }%
}
%    \end{macrocode}
% \end{macro}
%
%
% \begin{macro}{\filemod@gl}
%    \begin{macrocode}
\def\filemod@gl{>}
%    \end{macrocode}
% \end{macro}
%
%
% \begin{macro}{\Filemodcmp}
%    \begin{macrocode}
\newcommand\Filemodcmp[1][1]{%
    \def\filemod@next{\filemodcmpresult}%
    \Filemod@cmp{#1}%
}
%    \end{macrocode}
% \end{macro}
%
%
% \begin{macro}{\FilemodCmp}
%    \begin{macrocode}
\newcommand\FilemodCmp[1][1]{%
    \let\filemod@next\empty
    \Filemod@cmp{#1}%
}
%    \end{macrocode}
% \end{macro}
%
%
% \begin{macro}{\Filemod@cmp}
%    \begin{macrocode}
\def\Filemod@cmp#1#2#3{%
    \filemodgetnum{#2}%
    \let\filemoddatea\filemoddate
    \let\filemodtimea\filemodtime
    \filemodgetnum{#3}%
    \ifcase0%
        \ifnum\filemoddate>\filemoddatea\space1\else
            \ifnum\filemoddate=\filemoddatea\space
                \ifnum\filemodtime>\filemodtimea\space1\else
                    \ifnum\filemoddate=\filemoddatea\space2\fi
                \fi
            \fi
        \fi
    \relax
       \def\filemodcmpfile{#1}%
       \ifnum#1>2\relax
          \def\filemodcmpresult##1##2##3{##1}%
       \else
          \let\filemodcmpresult\@firstoftwo
       \fi
    \or
       \def\filemodcmpfile{#2}%
       \ifnum#1>2\relax
          \def\filemodcmpresult##1##2##3{##2}%
       \else
          \let\filemodcmpresult\@secondoftwo
       \fi
    \else
       \ifcase#1\relax
         \let\filemodcmpfile\empty
         \let\filemodcmpresult\@gobbletwo
       \or
         \def\filemodcmpfile{#1}%
         \let\filemodcmpresult\@firstoftwo
       \or
         \def\filemodcmpfile{#2}%
         \let\filemodcmpresult\@secondoftwo
       \else
         \let\filemodcmpfile\empty
         \let\filemodcmpresult\@thirdofthree
       \fi
    \fi
    \filemod@next
}
%    \end{macrocode}
% \end{macro}
%
%
% \iffalse
%</package>
% \fi
%
% \Finale
\endinput
